\begin{quote}
He waved a photograph of the Phaistos Disc. `The people who made this. Four thousand years ago! This was stamped! And if they were such pre-alfabetic genuises, then surely this must say something interesting, mustn't it? So I should be the first to read it, shouldn't I? The miserable thing is we have only this one specimen. But you don't make stamps for just one tablet, do you?'[\ldots]

`Let me explain my predicament to you.' He picked up a newspaper from the floor, and scribbled something in the margins.
`Write down the following number: eighty-three billion one hundred and ninety-one million twenty-four thousand five hundred and sixty-seven.' And after she had written it down on the sheet she used for her notes: `Now imagine an aboriginal cryptographer in an ancient Australian forest, who doesn't even know that those are numbers; al he sees is eleven incomprehensible marks: 8 3 1 9 1 0 2 4 5 6 7. They're all different, except the repeated symbol 1. Wat can he conclude from this? Nothing at al. That's the point I'm at now. Imagine that he suddenly has the brilliant idea that they're numbers, how is he supposed to figure out that they're alphabetically ordered, Dutch numerals from one to ten? How is he supposed to figure out that ``acht'' is the name of the number 8? He doesn't even know the decimal system, let alone the Dutch language.\\
---\emph{The Discovery of Heaven}, Harry Mulish \cite{mulisch1996discovery} 
\end{quote}


In the early chapter of the novel \emph{The Discovery of Heaven}; the magnum opus of Dutch author Harry Mulisch, we meet Onno Quist: one of the main characters. He is a disorgaized left-wing philologist, who has become obsessed with the \emph{Phaistos disc}: a mysterious artifact discovered centuries ago in the ruins of the Minoan palace on Crete. The disc contains a spiralling sequence of 242 symbols, from an alphabet of 45 distinct signs. Quist is convinced the disc contains an important message, and is sufficiently lacking in humility to decide that he must decipher it. But the Phaistos disc is the only example of a message of its script. 

\index{The Discovery of Heaven} \index{Harry Mulisch} \index{Phaistos Disc}

Thus, Quist is living a scientist's nightmare. Not for nothing are studies with small sample sizes easily dismissed as near-meaningless. To make strong inferences, hard claims, we cannot do without many repetitions: large amounts of examples of the same tyope of thing, over and over again. The quest for higher and higher sample sizes is the hallmark of hard science. The recent announcement of the discovery of the Higgs Boson, inferred from experimental data with a the famously stringet $5\sigma$ level of statistical significance, required 300 trillion repetitions of the proton collision experiment the Large Hadron Collider was built for. For a particle physicist, attempting anything with just one sample, must feel like digging a railway tunnel with a toothpick.

\index{Higgs Boson} \index{$5\sigma$} \index{Large Hadron Collider}

While we must feel sympathetic to Onno Quist's frustration, there is still is considerable difference between one sample and no samples at all. Consider this classic scenario, used the world over to instruct students in the basics of statistics. A soldier for a nation engaged in a bitter ground war is debriefed after being rescued from behind enemy lines. He relates to his superiors that he witnessed a new type of tank in a training exercise. The tank is clearly a secret weapon, and far in advance of anything their side can produce. The officers become anxious and wish to know how many of these tanks the enemy posess. The soldier only saw one, but he can relate that its serial number was 17.

\index{German tank problem}

Assuming that the enemy number their tanks in sequence, and that the soldier was as likely to spot this tank as any other, what does this one observation tell us about the number of tanks produced? Clearly, there are at least 17, but how many beyond that? What would be a good guess? One approach would be to take the number of tanks $n$ for which this outcome is most likely. For $n$ below 17, this outcome is impossible, and for higher $n$ the probability of any value is $1/n$. Thus, by this criterion, our best guess is that the soldier just happened to stumble on the tank with the highest serial number. This seems like a somewhat optimistic conclusion, so perhaps we need to adjust our criterion.

So instead of choosing the $n$ for which our particular outcome is the most likely, let us try to find a different procedure: one that (a) ensures that the average of our guesses if we were to repeat this exercise many times (one soldier, one observations, one guess) converges to the true value, and (b) produces the minimal expected distance between our guess and the true value. such a procedure is known as n unbiased minimum-variance estimator. For this problem, one exists, and it tells us to expect that the enemy has 33 tanks. It also tells us that if we want to make a statement with 95\% confidence, we should assume only that the enemy has between 17 and 340 tanks. Still a great deal of uncertainty, but much less than we would have had if we had dismissed the single sample as useless. This is not just an exercise to stimulate the minds of young students. During the second world war, the allies used exactly this procedure to estimate the number of Mark V tanks the Germans were producing (although in this case, they used more than one observation) \cite{davies2006statistical}.

This exercise is not much help to Onno Quist. The Phaistos disc contains no serial number, or at least not one that we can read. And if it did, the production capability of the ancient Minoans is of no more help in translating the contents of the disc, than the tank's serial number would be in guessing the names of its occupants. But Onno does have one advantage that the beleaguered nation didn't have: the Phaistos disc contains \emph{internal structure}. We can count its symbols, look at their frequencies, their proximities and co-occurrences. The number 17 could not be cracked open in this way to to find more information.

And in this respect Onno is not so lonely in quest. While few scientists are asked to compute the mean life expectancy, or the median income of a population from just one sample, when the objects under study become more complex and richly structured, the number of examples usually drops. Take the current web-graph, for example: the densely connected web of links between webpages on the internet. This is one of the most important ``objects'' of study in the world today. And like Onno, students of the webgraph have only one example to work with. 



